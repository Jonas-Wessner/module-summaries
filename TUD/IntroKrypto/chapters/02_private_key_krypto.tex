\chapter{Private-Key Cryptography}

\section{One Time Pad}

The One Time Pad works as follows:

\begin{enumerate}
    \item Alice has a message $m \in \{0,1\}^n$ to be sent to Bob. Furthermore, Alice and Bob possess a common key $k \in \{0,1\}^n, n\in N$. $n$ is called the security parameter.
    \item Alice computes a cipher text $c \in \{0,1\}^{n}$ as $c = m \oplus k$ and sends it to Bob.
    \item Bob reconstructs the plain message as $m = c \oplus k = m \oplus k \oplus k = m \oplus \{0\}^{n}$.
\end{enumerate}

Therefore, the abstract object for this cryptographic process consists of the following functions\footnote{The symbol $\perp$ is indicating an error.}:

\begin{itemize}
    \item $kGen(1^{n}) \rightarrow k \in \{0,1\}^{n}$
    \item $enc(m, k) \rightarrow c, \quad |m| = |k| = |c| = n$
    \item $dec(c, k) \rightarrow m \quad || \quad \perp$
\end{itemize}

The functional correctness is then defined as follows:\\
For all security parameter $n \in N$, for all messages $m$, for all keys $k \leftarrow kGen(1^{n})$, for all keys $k \leftarrow kGen(m,k)$ and for all ciphertexts $c \leftarrow enc(m,k)$ applies $dec(c,k) = m$. That means that we must be able to decrypt all encrypted messages for all possible input parameters.

\subsection*{Security of Shannon's OTP}

Independently of a bit $m_i$ in the message $m$, the OTP flips this bit with the probability of $\frac{1}{2}$ creating a distribution of ciphtertexts $c$ that is independent of the messages $m$. And because $m$ and $c$ are independent, it is intuitive that knowing $c$ cannot reveal anything about $m$. We can also prove this mathematically because the probability for two independent events to take place is the product of the individual probabilities: $Pr[M=m | C=c] = \frac{Pr[M=m \wedge C=c]}{Pr[C=c]} = \frac{Pr[M=m] \cdot Pr[C=c]}{Pr[C=c]} = Pr[M=m]$.