\chapter{Private-Key Cryptography}

\section{One Time Pad}

The One Time Pad works as follows:

\begin{enumerate}
    \item Alice has a message $m \in \{0,1\}^n$ to be sent to Bob. Furthermore, Alice and Bob possess a common key $k \in \{0,1\}^n, n\in N$. $n$ is called the security parameter.
    \item Alice computes a cipher text $c \in \{0,1\}^{n}$ as $c = m \oplus k$ and sends it to Bob.
    \item Bob reconstructs the plain message as $m = c \oplus k = m \oplus k \oplus k = m \oplus \{0\}^{n}$.
\end{enumerate}

Therefore, the abstract object for this cryptographic process consists of the following functions\footnote{The symbol $\perp$ is indicating an error.}:

\begin{itemize}
    \item $kGen(1^{n}) \rightarrow k \in \{0,1\}^{n}$
    \item $enc(m, k) \rightarrow c, \quad |m| = |k| = |c| = n$
    \item $dec(c, k) \rightarrow m \quad || \quad \perp$
\end{itemize}

The functional correctness is then defined as follows:\\
For all security parameter $n \in N$, for all messages $m$, for all keys $k \leftarrow kGen(1^{n})$, for all keys $k \leftarrow kGen(m,k)$ and for all ciphertexts $c \leftarrow enc(m,k)$ applies $dec(c,k) = m$. That means that we must be able to decrypt all encrypted messages for all possible input parameters.

\subsection*{Security of Shannon's OTP}

Independently of a bit $m_i$ in the message $m$, the OTP flips this bit with the probability of $\frac{1}{2}$ creating a distribution of ciphtertexts $c$ that is independent of the messages $m$. And because $m$ and $c$ are independent, it is intuitive that knowing $c$ cannot reveal anything about $m$. We can also prove this mathematically because the probability for two independent events to take place is the product of the individual probabilities: $Pr[M=m | C=c] = \frac{Pr[M=m \wedge C=c]}{Pr[C=c]} = \frac{Pr[M=m] \cdot Pr[C=c]}{Pr[C=c]} = Pr[M=m]$.


\section{Practical security}

In practice, perfect security is not feasible because this would mean that for transferring each message $m$ a key $k$ with at least the same length $|k| \geq |m|$ would have to be transferred (see definition of perfect security in section \ref{sec:gloss:perf_sec}). For this reason, we define a weakened security definition, which comes in two variants \textendash{} Concrete security and asymptotic security. In this course, we look at asymptotic security.

\begin{tabular}{|p{0.47\linewidth}|p{0.47\linewidth}|}
    \hline
    \textbf{Concrete Security:}                                                                                                       & \textbf{Asymptotic Security}                                                                                                                                          \\
    \hline
    A process is $(t,\mathcal{E})$-secure if no attacker $A$ can break it with at most $t$ steps with a probability of $\mathcal{E}$. & A process is secure, if no efficient (polynomial limited by security parameter $n$) algorithm breaks it with non-negligible (less than $\frac{1}{poly}$) probability. \\
    \hline
\end{tabular}

The relevant terms are defined as follows, where $n$ is the security parameter (e.g. key size):

\begin{itemize}
    \item \textit{Efficient Algorithm:} Algorithm with polynomial runtime with regard to $n$.
    \item \textit{Non-negligible Probability:} An inversely polynomial probability ($\frac{1}{poly}$) with regard to $n$.
    \item \textit{Negligible Probability:} Smaller than an inversely polynomial function (i.e. less than non-negligible) $\leq \frac{1}{poly}$. Mathematically speaking, a function $\mathcal{E} \coloneqq N \rightarrow R$ if there can be found a value $limit \in N$, such that for all polynomial functions $poly$ applies $\mathcal{E}(n) \leq \frac{1}{poly} \quad|\quad n \ge limit$. This means that, if the security parameter is high enough, its value is smaller than any polynomial function.
\end{itemize}

\subsection*{Examples for Negligibility of functions}

\begin{itemize}
    \item $\mathcal{E} = 2^{-n}$ is \textit{negligible} because exponential functions grow faster than polynomial functions.
    \item $\mathcal{E} = n^{-5}$ is \textit{not negligible} because there are polynomials that have smaller values as n approaches infinity e.g. $n^{-6}$.
    \item $\mathcal{E} = \begin{cases}
                  2^{-n} & \text{if $n$ is even}   \\
                  n^{-5} & \text{if $n$ is uneven}
              \end{cases}$
          is \textit{not negligible} because for some values (all uneven values) it behaves like a polynomial function, such that e.g. $n^{-6}$ would have greater values regardless of a chosen $limit$.
    \item $\mathcal{E} = \frac{1}{8}$ is \textit{not negligible} because it does not approach zero and therefore there are many polynomial functions that are smaller than $\mathcal{E}$ for high $n$
\end{itemize}

% TODO: continue in video 3 at 13:00