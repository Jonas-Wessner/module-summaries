\chapter{Public-Key Cryptography}\label{ch:pub_key_crypto}

%------------------------------------------
\section{Introduction}\label{sec:pubkey:intro}
%------------------------------------------

Public-key methods use the following functions for encrypting and decrypting:

\begin{align*}
    KGen(1^n)  & \rightarrow (sk, pk) \\
    Enc(pk, m) & \rightarrow c        \\
    Dec(sk, c) & \rightarrow m        \\
\end{align*}

Often, the public key can be computed efficiently from the secret key $f(pk) \rightarrow sk$.
But for security, it is required that the reverse of this function $f^{-1}$, which computes the secret key from the public key is \emph{not} efficiently computable.
We call such a function, that can only be efficiently computed in one direction a \emph{one-way function} (OWF).

%----------------------------------------
\section{One-Way Functions}\label{sec:owf}
%----------------------------------------

A one-way function (OWF) is a function $f: \{0,1\}^* \rightarrow \{0,1\}*$ that can be efficiently computed but its inverse function cannot be efficiently computed.

Formally speaking, a function $f$ is one-way if it has the following two properties:

\begin{enumerate}
    \item There is an efficient algorithm PPT for computing values $f(x)$ for all allowed input parameters $x$.
    \item For all efficient PPT algorithms A their success probability in the below security game $G^{inv}_{f,A}$ is negligible.
\end{enumerate}

$G^{inv}_{f,A}$ is defined as follows:

\begin{align*}
    \{0,1\}^\lambda & \$\rightarrow x \\
    f (x)           & \rightarrow y   \\
    A(1^\lambda, y) & \rightarrow x'  \\
    return f(x) == x'
\end{align*}

\textbf{Note 1:} This game does not require the adversary $A$ to find the value $x$ used but merely requires it to find a value that produces the output for $f$ i.e. a collision.

\textbf{Note 2:} This assumes that there are problems that are hard to compute (NP-hard). This is an assumption that holds so far but is not proven.

%---------------------------------------------
\section{Number Theory}\label{sec:number_theory}
%---------------------------------------------

\subsection{The Modulo Operator}

The modulo operator allows us to calculate with residual rings, which means that all values will never exceed the number $m-1$ for a residual ring $Z_m$.
We define addition and multiplication, such that they produce a value $x \in Z_m$ such that:

\begin{align*}
    a + b & = c mod m \quad |\quad \exists i \in Z: \quad a + b = c + i \cdot m     \\
    a + b & = c mod m \quad |\quad \exists i \in Z: \quad a \cdot b = c + i \cdot m
\end{align*}

\subsection{Groups}

A group is a combination $(G,\circ)$ of a set $G$ and an operation $\circ$ with the following 4 characteristics:

\begin{itemize}
    \item Closure: $a \circ b \in G \quad | \quad \forall a,b \in G$
    \item Associativity: $a \circ (b \circ c) = (a \circ b) \circ c \quad \forall a,b,c \in G$
    \item There exists an identity element (neutral element) $n$ such that $a \circ n = n \circ a = a \quad \forall a \in G$
    \item There exists an inverse element $a^{-1}$ for all elements $a$ in $G$ that can turn $a$ into the neutral element if combined with the operation $\circ$: $\exists a^{-1} \in G: \quad a \circ a^{-1} = a^{-1} \circ a = n \quad \forall a \in G$.
\end{itemize}

We call a group an abelian group if it is a group for which the operation $\circ$ over $G$ is commutative for all elements in $G$:
$a \circ b = b \circ a \quad \forall a,b \in G$.

Groups, or abelian groups, are nice to work with in cryptography.
Unfortunately, residual rings $Z_m$ are generally not groups if paired with the multiplication operator $(Z_m,\cdot)$ because they do not have inverse elements for all elements in $Z_m$. For example, each group $Z_m$ contains the value $0$, but no number can be multiplied by $0$ to produce the neutral element, which is $1$.
Additionally, depending on $m$, other elements might have no inverse element as well.

For this reason, we would like to reduce the set $Z$ to a set $Z_m^*$ that only contains all invertible elements of $Z_m$, thus making it a group.
The invertible elements of $Z_m$ are exactly all elements for which the greatest common divisor ($gcd$) is equal to $0$.
For instance, for the residual ring $Z_6 = \{0, 1, 2, 3, 4, 5\}$, the reduced group would be $Z_6^* = \{1, 5 \}$.

The number of elements in $Z_m^*$ is well defined for a couple of cases for prime numbers $p,q$ and positive integers $k$:

\begin{itemize}
    \item $|Z_p^*| = p-1$ (all elements of $Z_m$ except $0$)
    \item $|Z_{p^k}^*| = p^k - \frac{p^k}{p}$ (all elements except all multiples of p in $Z_{p^k}$)
    \item $|Z_{p \cdot q}^*| = |Z_{p}^*| \cdot |Z_{q}^*|$\footnote{This can be proven as follows: $\Phi(p \cdot q) = p \cdot q - (p-1) - (q-1) - 1 = p \cdot q - (p-1) - q  + 1 - 1 = (p-1)q - (p-1) = (p-1) \cdot (q-1) = \Phi(p) \cdot \Phi(q)$. These are all elements contained in $Z_{p,q}$ except all multiples of $p$ (there are $q-1$ of those and $q$ (there are $p-1$ of those). And also the element $0$ has to be subtracted.}
\end{itemize}

\textit{Def:} The \emph{order} of a group $(G, \circ)$ is the number of the elements in $G$: $ord(G) = |G|$.

\textit{Def:} The \emph{order} of an element $a \in G$ of a group $(G, \circ)$ is the number of elements that can be generated by applying the operation $\circ$ to $a$ and $a$ in $G$.

\textit{Def:} A group $(G, \circ)$ is \emph{cyclic} there exists an element $g \in G$, which we call the generator of the group, that can generate all elements of the group. This means that $ord(g) = ord(G)$.
It can be proven all groups $(Z_p^*, \cdot)$ with prime numbers $p$ are cyclic.

In cryptography, we would like to have a group $(G, \cdot)$ that is cyclic \emph{and} has a prime number of elements.
If we choose $Z_p^*$ with a prime $p$, we have a cyclic group. But the group has $(p-1)$ elements, which itself is not a prime\footnote{Since all prime numbers $>3$ are uneven numbers, $p-1$ for $p>2$ is even, therefore divisible by two and, hence, not prime.}.
To find a group that is cyclic and has a prime order, we first find a group $(Z_p^*)$ for a prime $p = wq + 1$\footnote{Example: $11 = 2 \cdot 5 + 1$, $w$ can be any positive integer.}, where $p$ and $q$ are both primes.
Since $p$ is prime, there is a generator $a$ for $Z_p^*$.
We can now find a subgroup with the desired characteristics by finding a generator $g$ that generates a subgroup $<g>$ of $Z_m^*$ that has a prime number of elements.
We do that by defining the generator as $g := a^w mod p$.
$g$ is a generator that generates all $w$-th elements of the initial group because if it is multiplied with itself it skips multiples of $a$, e.g. $g^3$ is equivalent to $(a^w)^3 = a^3w$ and $a^1,a^2$ are skipped.
This means that a group $<g>$ generated by $g$ has one $w$-th of the elements that the initial group $Z_m^*$ had:
It follow that: $|<g>| = \frac{|Z_m^*|}{w} = \frac{wq + 1 - 1}{w} = q$.
Thus, the order of $<g>$ is also prime (because we defined $q$ as a prime earlier).

%----------------------------------------------
\subsection{Fermat's Little Theorem}\label{sec:little_fermat}
%----------------------------------------------

Fermat's little theorem states that for any positive integer $a$ and for any prime number $p$ the following holds:

$$
    a^p \equiv a \; mod \; p
$$

If the greatest common divisor of $a$ and $p$ is 1\footnote{Which is true for at least all $a < p$ because $p$ is prime}, it also follows that:

$$
    a^{p-1} = 1 \; mod \; p
$$

because if we multiplied both sides by $a$, we would end up at the original equation.

%----------------------------------------------------
\section{Discrete Logarithm Problem}\label{sec:discrete_log}
%----------------------------------------------------

Let $(G, \cdot)$ be a group with prime order $q$, $g$ be a generator for the group and further $y \in G$ an element in the group.
The logarithm problem describes the task to find the (smallest) exponent for which $g$ generates $y$:

$$
    g^x \equiv y \; \text{in} \; G
$$

While solving this in the set of real numbers $R$ is easy, it is not easy for some groups with prime order.

The discrete logarithm assumption, which means that the discrete logarithm is hard to compute, is given for a group generator algorithm $Gen(1^\lambda)$ if all PPT algorithms win the below-defined game $Exp^{DL}_{Gen,A}$ only with negligible probability: $Pr[Exp^{DL}_{Gen,A}(1^\lambda) = 1] = neg(\lambda)$.

The game $Exp^{DL}_{Gen,A}(1^\lambda)$ is defined as follows:

\begin{align*}
    (G, q, g) & \leftarrow Gen(1^\lambda)           \\
    x         & \leftarrow^\$ \; G                   \\
    y         & \leftarrow g^x \; \text{in} \; G    \\
    x'        & \leftarrow A(1^\lambda, G, q, g, y) \\
    \text{return} \; [g^x \equiv y \; \text{in} \; G ]
\end{align*}

