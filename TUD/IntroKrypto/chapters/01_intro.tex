\chapter{Introduction}

Cryptography is the science of the processes for securing information, data and systems.

Typically we talk about certain key concepts, which differ from book to book. In this course, we define them as the following:

\begin{enumerate}
    \item[\textbf{C}] Confidentiality: Attackers cannot read a message.
    \item[\textbf{I}] Integrity: Attackers cannot modify the content of a message (in a narrow sense), cannot fake the message's sender address (authenticity) and receivers cannot claim, they have received the message (non-repudiation).
    \item[\textbf{A}] Availability: A system is always functional (less relevant in this course)
\end{enumerate}


Often in this lecture, we take a three-step approach to cryptographic processes:

\begin{enumerate}
    \item \textbf{Abstract Object:} We define what interface our cryptographic process should work with. Often, we define one process for encryption, one for decryption and maybe others like one for key generation. We do not yet define their implementation.
    \item \textbf{Security Model:} We define how our abstract object is used and in which cases our abstract object is secure. Questions to be answered are e.g. when is an attack successful and what is information the attacker is allowed to possess without introducing insecurity?
    \item \textbf{Security Proof a Specific process:} When given a concrete implementation of our security model, we check the security of that model. For example, we prove that an attacker must calculate at least $n$ hash sums to compromise the system.
\end{enumerate}

Cryptographic processes can be classified in a matrix as follows:

\begin{tabular}{|l||l|l|}
    \hline
                             & \textbf{Private-Key}                & \textbf{Public-Key} \\
    \hline\hline
    \textbf{Confidentiality} & Symmetric processes                   & Asymmetric processes  \\
    \hline
    \textbf{Integrity}       & Message Authentication Codes (MACs) & Digital signatures  \\
    \hline
\end{tabular}

Furthermore, there are elementary processes serving as building blocks to build cryptographic processes:

\begin{itemize}
    \item Pseudo random number generators (PRNGs)
    \item Hash functions
    \item Blockcipher
    \item Number theory
\end{itemize}