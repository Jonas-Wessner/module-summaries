\section{Introduction to Systems Programming}

\subsection{What is Systems Software?}

Important aspects of systems software:

Systems software ...

\begin{itemize}
    \item ... closely interacts with the hardware.
    \item ... is concerned about efficiency.
    \item ... is used by other software as opposed to application software which is used by the end-user directly.
\end{itemize}

Examples for systems software:

\begin{itemize}
    \item Operating system
    \item Compiler
    \item Game engine
    \item Search engine
    \item Programming languages virtual machines e.g. java virtual machine
    \item Device drivers
\end{itemize}

Examples for application software:

\begin{itemize}
    \item Text editor
    \item Shopping website
    \item Social media apps
    \item Chat client
\end{itemize}

\subsection{Systems Programming Languages}

Systems programming languages are languages which make systems programming easy. There are three properties we are especially interested in:

\begin{enumerate}
    \item Direct access to hardware resources:
          \begin{itemize}
              \item Memory management
              \item Network throughput
              \item GPU
              \item CPU (single or multi core)
              \item threads and processes
          \end{itemize}
    \item Performance, therefore mostly compiled languages
    \item It would be nice to have some useful abstractions to improve productivity (C/C++ $\rightarrow$ Rust or Go).
\end{enumerate}

Examples for systems programming languages:

\begin{itemize}
    \item C, C++
    \item Rust
    \item Go
    \item Assembly (rarely)
\end{itemize}

Examples for application programming languages:

\begin{itemize}
    \item JavaScript (disgusted tone of voice)
    \item Python
    \item Java
\end{itemize}