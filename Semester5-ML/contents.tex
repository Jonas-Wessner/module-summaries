\section{Metrics for Evaluating predictions}

\subsection{Confusion Matrix}
\subsection{Precision}
\subsection{Recall}
\subsection{F1 Score}
\subsection{Importance of the metrics}

\section{One-hot encoding}

\section{Overfitting and underfitting}

\subsection{How can it be detected?}

\subsection{Possible solutions}

\section{PCA - principal component analysis}

\subsection{Reasons for using PCA}
\subsection{Selection of good values for compon}

\section{Python Basics}

\subsection{Slicing}
\subsection{Data Extraction with Pandas}

\section{Regularization}
\subsection{What is regularization}
\subsection{Lasso}
\subsection{Ridge}
\subsection{Dropout}

\section{Machine Learning Tasks}

\subsection{Classification}
\subsection{Regression}
\subsection{Clustering}

\section{MLP - Multi-Layer-Perceptron}
\subsection{What is MPL?}
\subsection{Calculation of a number of parameters with and without bias}

\section{Feature map calculation in convolutional NN}

\section{Input and output sizes in Neural networks}
Describe here: Size of inputs and outputs in MLP and convolutional NN calculated from image size and the number of output classes.

\section{Activation functions}
\subsection{Softmax}
\subsection{Sigmoid}
\subsection{RELU}

\section{Solving non-linear problems with NNs}
Use example of logical function XOR here.

\section{K-means}

\section{Gradient Descent}

\section{Hyperparameters of ML models}

\subsection{Learning Rate}
\subsection{Epochs}
\subsection{Regularization}
\subsection{Batch Size}
\subsection{Convolution Kernel size}
\subsection{Max-Pooling}

\section{Logistic Regression and Cross Entropy}


\section{Linear Regression and Normal Equation}


\section{Decision Trees}


\section{K-nearest Neighbors}

